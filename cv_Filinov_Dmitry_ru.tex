\documentclass[10pt,a4paper]{article}
\pagestyle{empty}
\usepackage[utf8]{inputenc}
\usepackage[russian]{babel}
\usepackage[a4paper,
            left=1.25cm,
            right=1.25cm,
            top=1.5cm,
            bottom=1.5cm,
            columnsep=1.2cm]{geometry}
\usepackage{ragged2e} 
\usepackage[
    colorlinks=true,  
    linkcolor=blue,  
    urlcolor=blue,  
    citecolor=blue  
]{hyperref}
\usepackage{enumitem}
\usepackage{tabularx}


\newcommand{\sechead}[1]{%
  \noindent\textbf{#1}\par
  \vspace{-8pt}
  \noindent\rule{\linewidth}{0.4pt}\par\vspace{0.1cm}
}


\newenvironment{indentedsec}
  {\begingroup\setlength{\leftskip}{2em}}
  {\par\endgroup}
\begin{document}

\begin{center}
  {\huge\rmfamily\bfseries ФИЛИНОВ ДМИТРИЙ}
\end{center}
\begin{center}
\href{tel:+79130642157}{+7 (913) 064 21-57} \textbar\ 
\href{mailto:dm.filinov@gmail.com}{dm.filinov@gmail.com} \textbar\
  \href{https://t.me/D1ffic00lt}{Telegram} \textbar\ 
  \href{https://leetcode.com/u/D1ffic00lt/}{Leetcode} \textbar\ 
  \href{https://github.com/D1ffic00lt}{GitHub}
\end{center}
Data Scientist с практическим опытом в области машинного обучения, анализа данных и разработки бэкенда. Специализируюсь на задачах компьютерного зрения и NLP, включая классификацию, обнаружение и проверку фактов. Имею опыт создания сложных решений и участия в соревнованиях и хакатонах по искусственному интеллекту.
\vspace{0.3cm}

\sechead{ОПЫТ РАБОТЫ}

\textbf{ООО "Государство детей"} \hfill \emph{Проект онлайн-стажировки}

	\textit{Проект по сегментированию лесных пожаров на видео с помощью БПЛА.}\hfill \emph{09.2023 – 01.2024}\vspace{0.2cm}
	
	\textbf{Сибирская генерирующая компания, Новосибирская ТЭЦ-2} \hfill \emph{Некоммерческий проект}
	
\textit{Проект по созданию набора моделей ИИ для прогнозирования расхода топлива.} \hfill \emph{03.2023 – 07.2023}\vspace{0.2cm}

\vspace{0.3cm}

\sechead{НАВЫКИ}

\begin{tabular}{@{}l p{0.75\textwidth\qquad}@{}}
\textbf{Специализации} & Big Data, Классификация/сегментация/детекция изображений/видео, NLP \\
\textbf{Deep Learning}   & PyTorch, TensorFlow \\
\textbf{Computer Vision}& OpenCV, Roboflow, Torchvision, YOLO \\
\textbf{Data Visualization}& Matplotlib, Seaborn, Plotly \\
\textbf{Data Manipulation}& NumPy, Pandas, PyTorch \\
\textbf{MLOps}           & ClearML, Docker, Docker Compose, Poetry \\
\textbf{Backend}         & Flask, FastAPI \\
\textbf{Базы данных}       & SQL, SQLAlchemy, S3, Redis \\
\textbf{Прочее}           & Git, GitHub, \LaTeX \\
\textbf{Языки}&Русский (Носитель), Английский (Б2)
\end{tabular}
\vspace{0.3cm}

\sechead{ДОСТИЖЕНИЯ}

\textbf{DANO (2022 и 2023)} \hfill \emph{Всероссийский}

\textit{2022: 2-я степень}\hfill \emph{\href{https://github.com/D1ffic00lt/dano-olympiad-final-stage}{Репозиторий}}\vspace{0cm}

\textit{2023: Финалист}\hfill \emph{\href{https://github.com/D1ffic00lt/dano-2023}{Репозиторий}}\vspace{0.1cm}

\textbf{Кубок губернатора по компьютерному зрению} \hfill \emph{Региональный}

\textit{Команда: Победитель}\hfill \emph{\href{https://github.com/D1ffic00lt/computer-vision-cup}{Репозиторий}}\vspace{0cm}

\textit{Индивидуальный: 2 место}\vspace{0.1cm}


\textbf{Цифровой прорыв: Сезон ИИ 2023} \hfill \emph{Всероссийский}

\textit{2 место}\hfill \emph{\href{https://github.com/llitone/rutube-video-captioning}{Репозиторий}}\vspace{0.1cm}


\textbf{Цифровой прорыв: Сезон ИИ 2024} \hfill \emph{Всероссийский}

\textit{3 место}\hfill \emph{\href{https://github.com/D1ffic00lt/Sky-Eye}{Репозиторий}}\vspace{0.1cm}

\textbf{"Профессионалы"\ Региональный этап} \hfill \emph{Всероссийский}

\textit{2 место}\vspace{0.1cm}

\textbf{Другие соревнования (2021-2025)} \hfill \emph{Всероссийский}

\begin{itemize}[
    label=$\cdot$, 
    itemsep=0.1em,
    topsep=0pt,
    parsep=0pt
]
    \item \textit{Всероссийская олимпиада по искусственному интеллекту 2021} \hfill \emph{Топ 48}
    \item \textit{Всероссийская олимпиада по искусственному интеллекту 2024} \hfill \emph{Топ 27}
    \item \textit{Национальная технологическая олимпиада по искусственному интеллекту 2023} \hfill \emph{Топ 23}
    \item \textit{Национальная технологическая олимпиада по искусственному интеллекту 2024} \hfill \emph{Топ 14}
    \item \textit{Национальная технологическая олимпиада по Big Data и ML 2024} \hfill \mbox{\emph{Топ 8}\hspace{1ex}} 
    \item \textit{Data Fusion Contest — Задание 1 “Label Craft” 2025} \hfill \emph{Топ 14}
\end{itemize}\vspace{0.3cm}

\sechead{ОБРАЗОВАНИЕ}

\textbf{Национальный исследовательский университет "Высшая школа экономики"} \hfill \emph{Москва, Россия}

	\textit{Бакалавриат Прикладной Анализ Данных}\hfill \emph{09.2024 – настоящее}\vspace{0.2cm}
	
	\textbf{Московский физико-технический институт} \hfill \emph{Онлайн}
	
\textit{Deep Learning School, Базовый курс} \hfill \emph{09.2022 – 06.2023}\vspace{0.2cm}

\textbf{Yandex} \hfill \emph{Онлайн}

\textit{Интенсив по машинному обучению} \hfill \emph{2022}

\end{document}
